\chapter{Technical Issues}\label{chapter:technicalissues}
This chapter covers some of the problems encountered when turning the model from Chapter \ref{chapter:model} into code. This will not be an exhaustive list of all the challenges but will focus on the more interesting problems and solutions.

  \section{ABM frameworks}
    \paragraph{What is an ABM framework?}
    The first challenge was to create a software environment to run the simulation. 
    \paragraph{Why use an ABM framework? - Gilbert and Bankes paper}
    Gilbert and Bankes (\cite{Gilbert2002}) explain the advantages and
    disadvantages of using pre-existing libraries and frameworks for Agent-Based Models rather
    than ``rolling your own.'' Using pre-existing libraries frees a
    programmer up from re-implementing common
    algorithms. However, they also require a programmer to understand the
    language they are written in and to work with the built-in
    assumptions used by the original writers. Gilbert and Bankes state
    that agent-based modeling tools that match their ideal specifications
    do not yet exist but that there is a trend towards better
    tools becoming available. Fortunately a few years later Allen (\cite{Allan2009}) and Berryman (\cite{Berryman2008}) survey many ABM frameworks and suggest that NetLogo, MASON and Repast are useful frameworks. Personal communication with EngD student Ed Manley (\cite{Manley2012}) at UCL working on traffic simulations led me to look at the Repast framework. Therefore it seemed in 2012 with only a few months to complete this project but a good understanding of Java, using a pre-built framework would require less work than starting from the bottom and building a framework dedicated to the project.
    \subsection{ABM frameworks considered}
      \paragraph{Review of ABM frameworks - Allen and Berryman papers, hands}
      \subsubsection{Repast}
        \paragraph{descriptions of features/pros and cons of Repast}
        Open source
        Mailing list
        Documentation
        Examples from Repast City
        Java
        Random number generation
        Integrates nicely with Eclipse (so quick to get started)
        - slower than mason
        - less support for learning algorithms
      \subsubsection{NetLogo}
        \paragraph{descriptions of features/pros and cons of Netlogo}
        NetLogo requires the use of a proprietary
        language to program the agents aimed at beginner programmers, which I
        found too restrictive.
        Does not require much programming experience
        
      \subsubsection{MASON}
        Similar to RePast (in fact grew out of RePast according to )
    \subsection{The ABM framework I used}
      \paragraph{Why I chose ABM}
      Personal recommendation and one-to-one tutorial with a user. Berryman finds little to separate MASON and RePast so this is enough to swing it.
      Don't need to be able to handle a large number of agents (1000 boats fill the river)
  \section{Building the river}
    \subsection{Rastorization vs vectorization}
      \paragraph{Storing river as a series of squares/pixels}
      \paragraph{Storing river as a series of vectors (or nodes separated by vectors)}
      \paragraph{Breaking down river into 1D lanes adds benefits to vectorization}

  \section{Action execution}
    \subsection{Ensuring misuse of action does not break system}
      \paragraph{Making sure actions default to something sensible when they cannot be sensibly executed}
      \paragraph{What that default action is}
      
    \subsection{Ordering of cox actions and boat "reaction"}
      \paragraph{Ensuring that the cox cannot influence the boat too directly}
      \paragraph{Ensuring the boat's movement comes after the cox has execute his action to ensure response}
      \paragraph{Dealing with actions that take multiply timesteps}
      
  \section{Time discretization and scale}
    
    
  \section{creating a suitable visualization}
    \paragraph{motivation for visualization}
    \subsection{challenges}
      \paragraph{drawing the river}
      \paragraph{drawing the boat}
      
  \section{Experiment Framework}