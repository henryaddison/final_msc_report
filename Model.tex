\chapter{Model}
  \paragraph{introduction to how I went about forming the model}
  \section{Overview of model - introduce the model}
    \paragraph{Diagram of the model - can zoom to parts of it for following sections}
  \section{Objects}
    \paragraph{introducing the main objects that make up the universe of simulation}

    \subsection{River}
      \paragraph{Physical description}
      \subsubsection{Simplification from 3D space to into three 1D lanes}
        \paragraph{Reason for treating river as three parallel lines}
        \paragraph{Justification of 3 lanes}
        \paragraph{Network}
        \paragraph{Nodes}
        \paragraph{Edges}
        
      \subsubsection{Lock}
        \paragraph{description - where it is and whether boats can move past it}
        
    \subsection{Boathouse}
      \paragraph{description - starting and finishing place of boats and its location}
    
    \subsection{Cox}
      \paragraph{Description of duties and abilities}
      \subsubsection{Simplification for modelling}
        \paragraph{overall goal - get back to the boathouse}
      \subsubsection{How to model how the cox observes the river}
        \paragraph{River as a series of connected nodes to which a cox reacts as a model of cox's vision}
      \subsubsection{How to model how the cox observes other boats}
        \paragraph{Cox can record distance in terms of number of edges to next boat both in front and behind in the 3 lanes as a model of cox's vision}
        
    \subsection{Boat}
      \paragraph{Physical description}
      \subsubsection{Simplification for modelling}
        \paragraph{Speed in terms of gears and speed multiplier}
        

    
    \subsection{Outing Plans}
      \paragraph{description - relate it to boat speed, distance travelled and time taken}
      \paragraph{how they were modelled}
    
  \section{Decision making}
    \paragraph{introduction to what observation-decision-execution cycle}

    \subsection{Observations}
      \paragraph{introduction to observations as the information available to a cox}
      
      \subsubsection{Observations available to a cox}
      
    \subsection{Control Policy/Action Scheduling}
      \paragraph{description as the decision making process based on observations}
      \subsubsection{TR Program}
    
    \subsection{Actions}
      \paragraph{introduction to actions as the manner in which a cox has an effect on the environment}
      \subsubsection{Launching}
        \paragraph{describe - where launched, what lane}
      \subsubsection{Changing Speed}
        \paragraph{Describe how a cox alters speed through changing "gears"}
      \subsubsection{Letting Boat Run}
        \paragraph{description as default action - causes no change to the boat}
      \subsubsection{Changing Lane}
        \paragraph{describe - when can do it, where end up, what part of river it fills}
      \subsubsection{Spinning}
        \paragraph{describe - where can do it, what part of river it fills}
      \subsubsection{Landing}
        \paragraph{describe - where can land}
    
  \section{Movement}
    \subsection{Forward}
      \paragraph{forward movement and a boat's momentum - both physically and in sense that the rowers will continue until told to stop}
    \subsection{Steering}
      \paragraph{done automatically using information from nodes as a boat moves over them}
    \subsection{Crashing}
      \subsubsection{Defining in model context}
        \paragraph{Edge occupation}
        \paragraph{Relating lane edges to boat size}
      \subsubsection{Effects}
        \paragraph{description of real world}
        \paragraph{description in simulation}
        \paragraph{Why Moving slowly in a crash versus reversing - can ensure boats move monotonically forward}
        \paragraph{justification}
        \paragraph{why only let one boat move? - prevent boats staying in a permanently crashed state}
        \paragraph{justification}
  

