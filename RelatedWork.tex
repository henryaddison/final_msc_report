\chapter{Background Research/Related Work}
  \paragraph{Introductory paragraph explaining what similar works I looked at}
  

  \subsection{Scheduling problems solutions}
    \paragraph{How scheduling like for railways won't work - looking for a reactive solution not rigid scheduling problem solution}

  \section{Work done modeling and simulating river traffic}
    \paragraph{View boats as recreation river users - recreational as unlikely to cost the economy much money or elicit much sympathy from the world at large}

    \subsection{Lowry11 paper on using traffic simulation software}
      \paragraph{Description of paper contents}
      \paragraph{Similarities and differences to my own project}
      
  \section{Road traffic simulation}
      \paragraph{Introduction relating my problem to that of traffic simulation more commonly associated with road traffic. Restrict ourselves to microscopic models (and not interested in )}
      \subsection{Brief review of traffic simulation techniques}
        
        \subsubsection{traditional (aggregate) simulation methods transport modeling}
          \paragraph{Gipps' car following model}
          
          
        \subsubsection{Agent Based Models}
          \paragraph{Benefits of ABM - Bonabeau paper}
          \paragraph{Description of ABM traffic simulations}

          
        \paragraph{Conclusion about promising nature of multi-agent systems, especially on the  microscopic level as interested in effect of control policies on flow and safety safety}

        Ed Manley's reference to Nagel \& Flötteröd 2009
        
        From Ed Manley: Other authors note that vehicle units lack the ability to adapt or evolve in response to changes in an authentic manner (Andriotti \& Klugl 2006, Kotusevski \& Hawick 2009). This lack of flexibility means inclusion of other behavioural factors is more difficult, one important example being the impact of in-car travel information systems (or ATIS) on driver behaviour (Ben-Elia et al. 2007).
        It has also been noted through personal evaluation of the VISSIM software package (PTV AG 2010) that, in certain circumstances, vehicles may pass through each other at road crossings. This appears to represent a limitation of the car following model, which only accounts for those vehicles ahead of the individual. Where circumstances allow vehicles to cross in front of one another, the simulation software does not grant the autonomy to individuals that allow them to recognise and avoid this collision. This appears to demonstrate the significant limitations in regard to spatial perception on the part of the individual.

      \subsection{Why not using traffic simulators already around}
        \paragraph{based on road networks not small bit of river with no junctions}
        \paragraph{based on driver behaviour, not cox behaviour - no desire to get to destination quickly, may prefer sub-maximum speeds}
        \paragraph{Car are shorter than boats relative to speed}
        \paragraph{Boats accelerate slower and have longer response times}
  \section{Agent Based Modeling}
    \paragraph{View coxes as autonomous agents influencing the position of a boat on the river}
    \paragraph{Desire for any solution to be human readable - goal is to come up with suggestions for coxes NOT to replace them}
    \subsection{Uses for ABM}
      \paragraph{biological solution}
    
      
      \subsubsection{TR programs}
        \paragraph{Nilsson and Kochenderfer papers}
        
  