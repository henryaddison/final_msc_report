\chapter{Introduction}\label{chapter:intro}
  This chapter lays out what is to come in this project report. First it gives the motivation behind the work and the problem the work tries to solve. Then there is a short description of what I did and what I achieved over the last few months working on this project. Finally structure of this report is explained.

  \section{Motivation}
  My main motivation for this project was to create a simulation using an agent-based model. Rowing on a narrow and heavily used river like the Cam in Cambridge fits nicely into an agent-based model by treating the coxes as intelligent agents influencing the movements of boats, which themselves can be considered agents albeit not intelligent. I believe it is an interesting setting.

  Rowing crews competing in the Olympics need only contend with getting from start to finish of a straight 2km course where each boat is carefully segregated into its own wide lane.
  A cox of a boat on the Cam must deal with much more. He must steer along a narrow and
  winding path whilst avoiding the many other boats on the river, which will all be moving at different speeds and travelling both upstream and downstream. He will also
  have a coach on the bank shouting at him to follow a outing plan
  which dictates how fast and how far the boat should travel. And he
  must achieve all this with only the rudder to steer the boat and a few commands to the rowers to adjust the speed of the boat. For a new cox this is a lot to deal with. Wouldn't it be great if there were a few, simple rules of thumb he could follow?
  
  \section{The problem}
  The problem I set out to solve was to create a simulation of boats on the river Cam and use it to come up with a set of rules (a control policy) that someone relatively new to coxing on the Cam could follow in order to most of its time on the river. This simulation needs to have a visualization to help verify its correctness and so that it could be interpreted by a non-expert user. It also needs to output data in a more machine-friendly way so that the large volumes of data produced by batch runs of the simulation could be analysed.
  
  \section{What I achieved}
  After looking at the solutions to some similar problems, I designed an agent-based model of coxes and boats on a river. I then implemented this model in computer code with the help of Repast Simphony, an agent-based modelling framework written in Java. Repast Simphony made it easy to visualize the simulated boats as they moved about the river. I also put code in the simulation to write data to a database as it ran. I created some control policies in the form of Nilsson's teleo-reactive programs (\cite{Nilsson1994}). The simulation was run in a variety of scenarios to test these control policies I had devised. The data recorded from these simulation runs were then analysed to evaluate the performance of these control policies.
  
  I have learnt about agent-based modelling through creating one myself. I learnt about teleo-reactive programs by using them to create control policies for the cox agents. I also gained experience with a range of technologies when implementing this model and evaluating the control policies, including: the Java programming language and the Repast Simphony framework to create the simulation; SQL, MySQL databases and JDBC, the Java API for connecting to databases, to store and analyse data about simulation runs; the R software environment to plot graphs; and the \LaTeX typesetting system to create this report.
  
  \section{Report Stucture}
  This report follows an idealistic chronological path where background reading is done, a model designed, then a simulation implemented and used. The next chapter examines related work. Chapter \ref{chapter:model} describes the model. Chapter \ref{chapter:technicalissues} describes some of the challenges implementing this model in computer code. Chapter \ref{chapter:softwaredesign} describes the software that makes up the simulation, with Chapter \ref{chapter:control_policy} focussing on the logic behind the control policies created. Chapter \ref{chapter:experiments} describes the experiments with the simulation to evaluate the control policies. Finally, Chapter \ref{chapter:conclusions} gives my conclusions from the project, including recommending the best policy. I recommend reading them in the order they come.
